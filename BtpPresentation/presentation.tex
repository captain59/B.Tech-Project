\documentclass{beamer}
\usetheme{Boadilla}
%\usetheme{Bergen}
%\usetheme{Antibes}
\usepackage{caption}
\usepackage{amsmath}
\usepackage{tikz}
\usetikzlibrary{shapes.geometric, arrows, decorations.pathmorphing}
\tikzstyle{startstop} = [rectangle, rounded corners, minimum width=3cm, minimum height=1cm,text centered, draw=black, fill=red!30]
\tikzstyle{io} = [trapezium, trapezium left angle=70, trapezium right angle=110, minimum width=3cm, minimum height=1cm, text centered, draw=black, fill = red!30]
\tikzstyle{process} = [rectangle, minimum width=3cm, minimum height=1cm, text centered, draw=black, fill = orange!30]
\tikzstyle{decision} = [diamond, minimum width=3cm, minimum height=1cm, text centered, draw=black, fill = green!30]
\tikzstyle{arrow} = [thick, ->, >=stealth]
\setbeamertemplate{caption}[numbered]
\title{Cloud Removal in Images}
\subtitle{BTP Presentation}
\author{Anirban Mitra \\ 14EE10004}
\institute{IIT Kharagpur}
\date{\today}

\begin{document}
	\begin{frame}
		\titlepage
	\end{frame}
	\begin{frame}
		\frametitle{Outline}
		\tableofcontents
	\end{frame}
	\section{Section 1}
	\subsection{sub A}
	\begin{frame}
		\frametitle{Title 1}
		section 1 subsection A Title 1
		\begin{itemize}
			\item Point A
			\item Point B
			% Nested
			\begin{itemize}
				\item part 1
				\item part 2
			\end{itemize}
		\end{itemize}
	\end{frame}
	\begin{frame}
		\frametitle{Title 2}
		section 1 subsection A Title 2
		\begin{enumerate}[I]
			\item Point A
			\item Point B
			% Nested
			\begin{enumerate}[i]
				\item part 1
				\item part 2
			\end{enumerate}
		\end{enumerate}
	\end{frame}
	\subsection{sub b}
	\begin{frame}
		\frametitle{Using Columns}
		\begin{columns}
			\column{0.5\textwidth}
			Testing Out columns
			\column{0.5\textwidth}
			\centering
			\begin{figure}
				\includegraphics[scale=0.8]{images/kgplogo}
				\caption{IIT KGP LOGO}
			\end{figure}
		\end{columns}
	\end{frame}
	\section{Section 2}
	\begin{frame}
		\frametitle{Pictures Figures}
		\begin{figure}
			\includegraphics[scale=1.2]{images/kgplogo}
			\caption{IIT KGP LOGO}
		\end{figure}
	\end{frame}
	\begin{frame}
		\frametitle{Descritption}
		\begin{description}
			\item[API] Application Programming Interface
			\item[LAN] Local Area Network
			\item[ASCII] American Standard Code for Information Interchange
		\end{description}
	\end{frame}
	\begin{frame}
		\frametitle{Adding Tables}
		\begin{table}[h]
			\begin{tabular}{l | c | c | l}
				Serial N.o & Name & Age & Sex \\
				\hline
				1 & Anirban & 22 & M \\
				\hline
				2 & Arunima & 24 & F
			\end{tabular}
			\caption{Simple Table}
			\label{tab:tri}
		\end{table}
	\end{frame}
	\begin{frame}
		\frametitle {Matrices}
		Here's a matrix \\
		\[
		\begin{bmatrix}
		1 & 2 & 3 \\
		4 & 5 & 6
		\end{bmatrix}
		\]
	\end{frame}
	\begin{frame}
		\frametitle{Block}
		\begin{alertblock}{Alert Block Title}
			This forms the content of the block
		\end{alertblock}
		\begin{block}{Block Title}
			This forms the content of the block
		\end{block}
		\begin{definition}{Definition Block}
			Content of Block
		\end{definition}
		\begin{example}
			Example Block
		\end{example}
	\end{frame}
	
	\begin{frame}
		\frametitle{Adding Code}
		\begin{semiverbatim}
			content
		\end{semiverbatim}
	\end{frame}
	\begin{frame}
		\frametitle{Creating Flowchart}
		\begin{tikzpicture}[node distance=2cm]
		\node (start) [startstop] {Start};
		\node ()
		\end{tikzpicture}
	\end{frame}
\end{document}